\documentclass[a4paper, 11pt, hidelinks]{article}
\usepackage[left=2cm, top=3cm, text={17cm, 24cm}]{geometry}
\usepackage[czech]{babel}
\usepackage[utf8]{inputenc}
\usepackage{hyperref}
\usepackage{tabularx}
\usepackage{booktabs}
\usepackage{xcolor}

\begin{document}


\begin{titlepage}
    \begin{center}
        \Huge \textsc{Vysoké učení technické v~Brně} \\
        \huge \textsc{Fakulta informačních technologií} \\
        \vspace{\stretch{0.382}}
        Síťové aplikace a správa sítí\,--\,monitorování DHCP komunikace\\
        \Huge  Manuál k programu dhcp-stats\\
        \vspace{\stretch{0.618}}
        \Large \today{} \hfill Ondřej Koumar (xkouma02)
    \end{center}
\end{titlepage}

\tableofcontents

\newpage

\section*{Úvod}\label{0_uvod}
\addcontentsline{toc}{section}{Úvod}
Ne vždy má uživatel možnost sledovat statistiku vytížení síťových prefixů.
Některé DHCP servery tuto statistiku poskytují samy, případně se mohou parsovat adresy z jejich logů.
Pokud ale takováto možnost není, nezbývá moc jiných možností, než sledovat DHCP provoz odchytáváním paketů a manuálním parsováním IP adres, které jsou klientským stanicím přidělovány.
To je use-case, pro který byl program \texttt{dhcp-stats} vytvořen.

Program \texttt{dhcp-stats} je schopen monitorovat DHCP provoz na vybraném rozhraní, ať už se jedná o ethernetové či bezdrátové, a generovat statistiku pro uživatelem zadaný síťový prefix.
Samotné monitorování spočívá v odchytávání a filtraci paketů, které na daném rozhraní projdou, a hledání podstatných informací v nich.
V paketu, který je odeslán DHCP serverem, případně klientem, který DHCP službu využívá, jsou umístěny veškeré informace, které jsou potřebné pro generování statistiky pro síťový prefix.

Pokud uživatel nemá potřebu přímo monitorovat DHCP provoz naživo, má možnost využít zpracování \emph{.pcap} nebo \emph{.pcapng} souborů (které si může vygenerovat například pomocí programu Wireshark\footnote{O programu se dozvíte na \href{https://www.wireshark.org/about.html}{https://www.wireshark.org/about.html}.}).
To se hodí například v případě, že má již síť zmonitorovanou a nepotřebuje real-time statistiku.
V tomto případě je ale počítat s tím, že možnosti programu jsou omezené; vypíše se pouze finální podoba sítě, tedy vytížení prefixů v době, kdy skončilo zachytávání paketů.
Je ale možné do programu uměle vložit zpoždění tak, aby šlo vidět zpracování paketů po jednom.
Více o této možnosti v kapitole \ref{3_popis}.

\newpage
\section{Uvedení do problematiky}\label{1_problematika}
DHCP (\emph{Dynamic Host Configuration Protocol}) je popsán v rámci \href{https://datatracker.ietf.org/doc/html/rfc2131}{RFC 2131}. 
DHCP se skládá za dvou částí\,--\,z protokolu zajišťujícího doručování parametrů od serveru ke klientovi a systém alokací IP adres.
Je postaven na modelu klient-server, kde server je zařízení, které poskytuje parametry přes přenosovou část protokolu a klient je zařízení, které o tyto parametry DHCP server žádá.

\subsection{DHCP pakety}\label{1_1_pakety}
Položky DHCP paketu a jejich vysvětlení:
\begin{table}[ht]
  \centering
  \begin{tabularx}{\textwidth}{p{0.1\textwidth}p{0.15\textwidth}p{0.67\textwidth}}
    \toprule
    \textbf{Pole} & \textbf{Velikost (B)} & \textbf{Popis} \\
    \midrule
    \texttt{op} & 1 & Kód operace zprávy. \\
    \texttt{htype} & 1 & Typ hardwarové adresy. \\
    \texttt{hlen} & 1 & Délka hardwarové adresy. \\
    \texttt{hops} & 1 & Počet relay agentů\footnotemark[2], přes které šla DHCP zpráva. \\
    \texttt{xid} & 4 & ID transakce, které používají jak server, tak klient k identifikaci zpráv. \\
    \texttt{secs} & 2 & Počet sekund od začátku transakce (žádost o adresu nebo obnovení \emph{lease time}.) \\
    \texttt{flags} & 2 & Příznak broadcastu. \\
    \texttt{ciaddr} & 4 & IP adresa klienta. \\
    \texttt{yiaddr} & 4 & Nová klientova IP adresa (použito při nabízení a potvrzování adres od serveru). \\
    \texttt{siaddr} & 4 & IP adresa serveru, který má klient kontaktovat při další zprávě. \\
    \texttt{giaddr} & 4 & IP adresa relay agenta, přes kterého se posílají DHCP zprávy. \\
    \texttt{chaddr} & 16 & MAC adresa klientovy síťové karty. \\
    \texttt{sname} & 64 & Jméno serveru (volitelné). \\
    \texttt{file} & 128 & Název souboru, který klient použije pro bootstrapping\footnotemark[3]. \\
    \texttt{options} & n & Pole volitelných parametrů.\\
    \bottomrule
  \end{tabularx}
  \label{tab:dhcp-format}
  \caption{Formát zprávy DHCP}
\end{table}

\footnotetext[2]{V kontextu DHCP se jedná o proces získání potřebných informací od DHCP serveru.}
\footnotetext[3]{Relay agenti mají na starost přeposílání paketů mezi klientem a serverem při DHCP komunikaci.}

Poznámky:
\begin{itemize}
    \item \texttt{op} Může nabývat hodnot BOOTREQUEST a BOOTREPLY.
    \item Výčet typů hardwarových adres \href{https://www.iana.org/assignments/arp-parameters/arp-parameters.xhtml#arp-parameters-2}{\textcolor{blue}{zde}}.
    \item \texttt{xid} je náhodné číslo generované klientem, používá se po celou dobu komunikace.
    \item \texttt{flags} je ve formátu: 

        \begin{table}[ht]
            \centering
            \begin{tabular}{|c|c|}
            \hline
            \textbf{Bits} & \textbf{Value} \\
            \hline
            0             & B \\
            1-15          & MBZ \\
            \hline
            \end{tabular}
            \caption{\texttt{flags} pole}
            \label{tab:dhcp_flags}
        \end{table}
        
        kde \emph{B} je příznak, že zpráva byla poslána na broadcastovou adresu sítě, \emph{MBZ} je 15 bitů, které jsou vždy nastaveny na 0.
    \item \texttt{options} je proměnné velikosti, klient ale musí být připraven přijmout DHCP zprávu, která má velikost \texttt{options} alespoň 312\,B.
    Ne všechny parametry se do pole \texttt{options} musí vlézt; pokud je třeba, DHCP server může využít pole \texttt{sname} případně pole \texttt{file} pro uložení zbývajících parametrů.
    Pro výčet parametrů, které se v tomto poli mohou vyskytnout, vizte \href{https://www.iana.org/assignments/bootp-dhcp-parameters/bootp-dhcp-parameters.xhtml#options}{\textcolor{blue}{zde}}.
\end{itemize}

\subsection{DHCP komunikace}\label{1_2_komunikace}

\section{Návrh aplikace}\label{2_navrh}

\section{Popis implementace}\label{3_popis}

\section{Základní informace o programu}\label{4_zakladni_info}

\section{Návod na použití}\label{5_navod}

\end{document}
