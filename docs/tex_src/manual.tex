\documentclass[a4paper, 11pt, hidelinks]{article}
\usepackage[left=2cm, top=3cm, text={17cm, 24cm}]{geometry}
\usepackage[czech]{babel}
\usepackage[utf8]{inputenc}
\usepackage{hyperref}

\begin{document}


\begin{titlepage}
    \begin{center}
        \Huge \textsc{Vysoké učení technické v~Brně} \\
        \huge \textsc{Fakulta informačních technologií} \\
        \vspace{\stretch{0.382}}
        Síťové aplikace a správa sítí\,--\,monitorování DHCP komunikace\\
        \Huge  Manuál k programu dhcp-stats\\
        \vspace{\stretch{0.618}}
        \Large \today{} \hfill Ondřej Koumar (xkouma02)
    \end{center}
\end{titlepage}

\tableofcontents

\newpage

\section{Uvedení do problematiky}\label{1_problematika}
Program \texttt{dhcp-stats} je schopen monitorovat DHCP provoz na vybraném rozhraní, ať už se jedná o ethernetové či bezdrátové, a generovat statistiku pro uživatelem zadaný síťový prefix.
Tato vlastnost se může hodit v situaci, kdy DHCP server tuto statistiku neposkytuje. 
Samotné monitorování spočívá v odchytávání a filtraci paketů, které na daném rozhraní projdou, a hledání podstatných informací v nich.
V paketu, který je odeslán DHCP serverem, případně klientem, který DHCP službu využívá, jsou umístěny veškeré informace, které jsou potřebné pro generování statistiky pro síťový prefix.

Pokud uživatel nemá potřebu přímo monitorovat DHCP provoz naživo, má možnost využít zpracování \emph{.pcap} nebo \emph{.pcapng} souborů (které si může vygenerovat například pomocí programu Wireshark\footnote{O programu se dozvíte na \href{https://www.wireshark.org/about.html}{https://www.wireshark.org/about.html}.}).
To se hodí například v případě, že má již síť zmonitorovanou a nepotřebuje real-time statistiku.
V tomto případě je ale počítat s tím, že možnosti programu jsou omezené; vypíše se pouze finální podoba sítě, tedy vytížení prefixů v době, kdy skončilo zachytávání paketů.
Je ale možné do programu uměle vložit zpoždění tak, aby šlo vidět zpracování paketů po jednom.
Více o této možnosti v kapitole \ref{3_popis}.

\section{Návrh aplikace}\label{2_navrh}

\section{Popis implementace}\label{3_popis}

\section{Základní informace o programu}\label{4_zakladni_info}

\section{Návod na použití}\label{5_navod}

\end{document}
